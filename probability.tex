

\paragraph{Sample Space and Outcome}
We perform random experiments and the sample space is the set of possible
outcomes.

For example, consider rolling a die. The set of possible outcomes are:
\begin{equation*} \begin{split}
	S = \{1,2,3,4,5,6\}
\end{split} \end{equation*}

\paragraph{Event} An event is a subset of the sample space. An example event is
rolling a die and getting an even odd outcome:
\begin{equation*} \begin{split}
	E = \{1,3,5\}
\end{split} \end{equation*}

\paragraph{Disjunction of Events} The event $E$ occurs if $E_1$ or $E_2$ occur.
Another way to imagine this is the union of events: $E = E_1 \cup E_2$.

\paragraph{Conjunction of Events} The event $E$ occurs if $E_1$ and $E_2$ occur.
Another way to imagine this is the intersection of events: $E = E_1 \cap E_2$.

\paragraph{Mutually Exclusive Events} Events $E_1$ and $E_2$ are mutually
exclusive if only one of them can occur in a single experiment. For example, the
event rolling an even number and the event rolling an odd number on a die are
mutually exclusive events:
\begin{equation*} \begin{split}
E_{even} \cap E_{odd} =  \{1,2,3,4,5,6\} \cap \{1,3,5\} = \varnothing
\end{split} \end{equation*}

\subsubsection{Axioms of Probability}
The are the rules we accept as truth without proof. We build probability untop
of these axioms.

\begin{enumerate}
	\item $0 \le P(E) \le 1$, for any event $E$. In the smallest case, the event
		cannot occur which is inidicated by a probability of 0. In the largest case,
		the event always occurs, which is indicated by the probability of 1.
	\item $P(S) = 1$, where S is the sample space. The sample space contains all
		possible outcomes for each experiment. It's reasonable to accept that an
		event from the sample space always occurs.
	\item For a potentially infinite set of mutually exclusive events $E_1$,
		$E_2$, ... 
		\begin{equation*} \begin{split}
			P( \cup_{i=1}^\infty E_i ) = \sum_{i=1}^\infty P(E_i)
		\end{split} \end{equation*}
		It makes senses that events that do not share outcomes for a single event,
		can have their probabilities added to arrive at the probability of combining
		the outcomes from the events.
\end{enumerate}

\paragraph{Properties} From the above axioms, we get the following useful
properties:
\begin{enumerate}
	\item 
	\item 
	\item 
\end{enumerate}

\begin{equation*} \begin{split}
\end{split} \end{equation*}

