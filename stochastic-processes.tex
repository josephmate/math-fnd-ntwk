
A family of random variables, indexed by time

\subsubsection{Classifications}

\paragraph{State Space} The set of possible values (states).
\paragraph{Discrete State Space} Example: the number of jobs in the system.
($S = {0,1,2,3...}$). We will only deal with the discrete case in this class. To
make notation easier the state is usually identified by the number.

\paragraph{Continuous State Space} Example: motion of a particle. We will not be
studying this in this class.

\paragraph{Time Parameter} There are two ways to observe times in Stochastic
processes.

\paragraph{Discrete Time Parameter} For example, we consider the states at
$X_0$, $X_1$, ... $X_n$. For example, looking at the state of the system
$i^{th}$ hour.

%TODO fix diagram
%

\begin{tikzpicture}[scale=0.5]
	\tkzDrawSegment({0,0},{3,0}) 
	\tkzDrawSegment({4,0},{7,0}) 
	\draw[dotted, line width = 0.5mm](3,0) -- (4,0);

	\foreach \i in {0,...,2}{ 
	\begin{scope}
		\tkzDrawSegment({\i,-1/2},{\i,1/2})
	\end{scope}
	}

	\foreach \i in {5,...,7}{ 
	\begin{scope}
		\tkzDrawSegment({\i,-1/2},{\i,1/2})
		\node
	\end{scope}
	}
\end{tikzpicture}


\paragraph{Continuous Time Parameter}
The states are function of time $t$ ($X(t)$).

\subsubsection{Discrete State Space and Time} There might be a dependency
between the previous time interval $X_i$'s and the states those time interval
can be in that need to be model in the current $X_n$. 
\begin{equation*} \begin{split}
	P(X_{n+1} = j | X_n = i_n\, , X_{n-1} = i_{n-1}\, , ... \, , X_1 = i_1\, ,X_0 = i_0)
\end{split} \end{equation*}
The number of dependency combinations is exponential because \\
$X_{n+1}$ depends on $X_n$ to $X_0$ \\
and $X_{n}$ depends on $X_{n-1}$ to $X_o$ \\
and so on. 

\paragraph{Markov Chain} As a result of the exponential size, we make a
simplifying assumption. We only use the latest information. $X_{n+1}$ only
depends on $X_n$. Now we are left with transition probabilities:
\begin{equation*} \begin{split}
	P(X_{n+1} | X_n) = P(X_{n+1} = j | X_n = i_n\, , X_{n-1} = i_{n-1}\, , ... \, , X_1 = i_1\, ,X_0 = i_0)
\end{split} \end{equation*}
Given $P(X_0 = i)$ for all $i$'s, we can compute any state. However, notice that
the formula depends on n, the discrete time that has pasted so far, which make
analysis difficult still. For example, in the 9am one hour interval the number
of jobs in a login system tends to be higher than at 2am.

\paragraph{Homogeneous Markov Chains} We now make the assumption that that the
transistion probabilities do not depend on time. For example, the transition
probabilies for the number of webcrawling robots requesting a webpage remain the
same despite the time. So we can write
\begin{equation*} \begin{split}
	P(X_{n+1} = j | X_n = i ) = P(X_{n} =j | X_{n-1} = i ), \forall n\,i\,j \ge 0
\end{split} \end{equation*}
Which is abbreviated to $P_{ij}$






\begin{equation*} \begin{split}
\end{split} \end{equation*}












